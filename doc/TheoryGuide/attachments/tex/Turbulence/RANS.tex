\subsection{Reynolds-Averaged Navier Stokes (RANS) Models}

Reynolds-Averaged Navier Stokes models are based on either solving for the mean flow using the eddy viscosity hypothesis, or close the Reynolds stress tensors, $u_iu_j$. At the moment, LUNA only implements eddy viscosity models. 

The turbulent viscosity hypothesis, or eddy viscosity hypothesis, assumes that the turbulent effects can be modeled via a dissipative eddy viscosity term, denoted $\nu_T = \mu_T / \rho$ in the Navier Stokes equations. Therefore, the key to solving the closure problem in eddy viscosity models is determining the eddy viscosity. Note that $\nu_T$ is a property of the flow, not the fluid (like the standard viscosity $\nu$).

\begin{equation}
\langle u_i u_j \rangle = \frac{2}{3} k \delta_{ij} - \nu_T \left[ \pdv{\overline{U_i}}{x_j} + \pdv{\overline{U_j}}{x_i} \right]  \label{eddyViscosityHypothesis}
\end{equation}

The eddy viscosity model makes two major assumptions:

\begin{enumerate}
\item The Deviatoric/anisotropic part of the Reynolds stress tensor is treated as a function of local mean strain rates only (i.e. in eq. \ref{eddyViscosityHypothesis}, LHS = RHS for all $i$, $j$ in isotropic turbulence)
\item The relationship between the Reynolds stress and the mean strain rates are linearly related by a single scalar
\end{enumerate}


\subsubsection{Spalart-Allmaras}

Spalart-Allmaras is a low-Reynolds number one equation RANS turbulence model. In other words, one transport equation is required to be solved to determine the eddy viscosity. For a general compressible flow, the eddy viscosity is given as 

\begin{equation}
\mu_T = \rho f_{v1} \tilde{\nu}
\end{equation}

\noindent where 

\begin{itemize}
\item $\rho$ is the density
\item $f_{v1}$ is a damping function 
\item $\tilde{\nu}$ is a modified diffusivity
\end{itemize}

\noindent The damping function is:

\begin{equation}
f_{\nu 1} = \frac{\chi^3}{\chi^3 + c_{\nu1}^3}
\end{equation}

\noindent where:

\begin{equation}
\chi = \frac{\tilde{\nu}}{\nu}
\end{equation}

\noindent The transport equation for the modified diffusivity $\tilde{\nu}$ is:

\begin{equation}
\pdv{\rho \tilde{\nu}}{t} + \pdv{\rho \tilde{\nu} u_j}{x_j} = \frac{1}{\sigma_{\tilde{\nu}}} \pdv{}{x_j} \left[ \left(\mu + \rho \tilde{\nu} \right) \pdv{\tilde{\nu}}{x_j} \right] + P_{\tilde{\nu}} + S_{\tilde{\nu}} \label{SA_transport}
\end{equation}

\noindent where:

\begin{itemize}
\item $u_i$ is the mean velocity
\item $\sigma_{\tilde{\nu}}$ is a model coefficient
\item $\mu$ is the dynamic viscosity
\item $P_{\tilde{\nu}}$ is the production term
\item $S_{\tilde{\nu}}$ is source term
\end{itemize}

\noindent Note that using the continuity equation, eq. \ref{SA_transport} is equivalent to:

\begin{equation}
\rho \left[ \pdv{\tilde{\nu}}{t} + u_j \pdv{\tilde{\nu}}{x_j} \right] = \frac{1}{\sigma_{\tilde{\nu}}} \pdv{}{x_j} \left[ \left(\mu + \rho \tilde{\nu} \right) \pdv{\tilde{\nu}}{x_j} \right] + P_{\tilde{\nu}} + S_{\tilde{\nu}}
\end{equation}

