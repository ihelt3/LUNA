%%%%%%%%%%%%%%%%%%%%%%%%%%%%%%%%%%%%%%%%%%%%%%%%%%%%%%%%%%%%%%%%%%%
%% SIMPLE
\subsection{Linear Algebra}




%------------------------------------------------------------------
\FloatBarrier
\subsubsection{Sparse Matrices}

In an implicit CFD solver, a linear system is formed and solved for all transport equations over every cell. This means a matrix must be created for each each of these systems, which has a size of $n \times n$, where $n$ is the number of cells in the mesh. Obviously, this matrix could quickly become very large and take up a huge amount of memory. However, in these matrices, there are a large number of 0 elements. This fact allows us to cleverly store matrix data to save on memory. The LUNA solver uses a Compressed Sparse Row (CSR) matrix format to save on memory, which is detailed here. 

The CSR matrix stores data in 3 different vectors, which are detailed below. NNZ refers to the number of nonzero elements in a general $m \times n$ matrix, where $m$ is the number of rows, and $n$ is the number of columns.

\begin{itemize}
\item $\vec{v}$: A vector which stores all the non-zero vectors in the matrix (size: NNZ)
\item $\vec{c}$: A vector which stores the column associated each value in the $\vec{v}$ vector (size: NNZ)
\item $\vec{r}$: A vector which always starts with 0, and stores the difference in NNZ elements before and after the row (size: $m+1$)
\end{itemize}

An example is given below for clarity:

\begin{equation}
A = \begin{bmatrix}
0 & 0 & 4 & 0 \\
1 & 0 & 0 & 2 \\
8 & 0 & 0 & 0 \\
0 & 0 & 0 & 0 \\
0 & 7 & 0 & 0
\end{bmatrix}
\end{equation}

\begin{align}
\vec{v} &= \begin{bmatrix} 4 & 1 & 2 & 8 & 7 \end{bmatrix} \\
\vec{c} &= \begin{bmatrix} 3 & 1 & 4 & 1 & 2 \end{bmatrix} \\
\vec{r} &= \begin{bmatrix} 0 & 1 & 3 & 4 & 4 & 5 \end{bmatrix}
\end{align}

The $\vec{v}$ vector is relatively self explanatory in the above example. The $\vec{c}$ vector starts with $3$, indicating the that the value of $4$ is in the $3^\mathrm{rd}$ column, followed by $1$ indicating that the value of $1$ is in the $1^\mathrm{st}$ column, and so on. 

The $\vec{r}$ vector is the least obvious. It starts with 0, as it always does, and then $1$ indicates there was 1 NNZ in the first row. Since there were 2 NNZ in the second row, the third value is $1 + 2 = 3$, where the $1$ is the NNZ in the previous rows. The fourth value is $4$ since there was 1 NNZ in the third row ($1+3=4$), and in the fourth row there are no NNZ, so the vector stays at $4$, finally in the last row there is one last NNZ, so the final value is $4+1=5$. 

While this method does not appear to save much memory for small matrices, it can have a huge impact on the amount of storage used for bigger matrices. 

